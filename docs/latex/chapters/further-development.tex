\chapter{Further avenues of development}

Over the course of the semester, the graphical section far outweighed the program's other capabilities and required the most development time. If I would ever start a similar project from scratch, I'd make use of a game engine such as Unity or Godot to handle the OpenGL or Vulkan calls on my behalf. I find the experience valuable, however. I understand the basic computer graphics concepts better now, and during my reading, I learned a lot about shader programming as well.

While the program is already capable of providing good insights into many cities' basic travel patterns, it may be improved in the future in multiple ways. These include:
\begin{itemize}
    \item a more robust networking system, capable of efficiently handling dropped packets and service disruptions -- this would require a medium-to-large scale refactoring of the code, and is feasible if any of the other features are implemented
    \item the addition of roads to the simulation model, making agents travel along these known good paths instead of the "as the crow flies" method
    \item different modes of transportation, including walking, car use and public transit
    \item adding real public transit lines with vehicles 
    \item interactive map with the ability to move and delete buildings or other objects
    \item fine-tuning the simulation by changing distribution parameters on the frontend
    \item general optimisations
\end{itemize}