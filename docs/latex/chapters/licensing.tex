\label{licensing}
\section{Licensing}

I strongly believe in free and open source software, so I wanted to pick a distribution framework that would allow other programmers to use or modify the program if they wish to do so. Because of this, I chose to use the GNU General Public License v3. Created by the GNU Foundation on 2007, the license permits anyone to make changes to, distribute, or commercialise a product. All changes must be stated and redistributed under this same scheme or a related one. Any software made using this project must be public -- this prevents its inclusion in proprietary, closed company projects. As the creator, I'm not legally required to provide any warranty for the software.\cite{choose-gpl3} The license declaration document was part of the GitHub repository I created, from the very start. In the future, I intend to keep the program and thesis document public.

The evident limitation caused by this choice is that the thesis project could not make use of any closed source library or tool. LibGDX is under Apache 2.0, which is entirely compatible with GPLv3. The fonts and icons used from Google's library\footnote{\url{https://fonts.google.com/}} use the SIL Open Font License, which permits anything short of taking ownership of and selling the work. OpenStreetMap data is distributed using the custom Open Data Commons Open Database License (ODbL). As long as credit is given (which the app's main scene provides), the data may be freely used or built upon.