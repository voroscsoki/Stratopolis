\chapter{The current state of the transit modelling industry}
The purpose of modelling public transit systems is two-fold: evaluating current service patterns and identifying new needs is of equal importance. Initial data collection is just as important as finding ways to visualise it. A well-rounded dataset can allow us to make predictions and conclusions about the whole of a city. In this section, I'll introduce data collection and modelling solutions. Wherever appropriate, I will also show examples pertaining to my home city, Budapest.

\section{Data collection}
\subsection{Traditional, in-person method}

Up until the 2000s, the main method of collecting information was passenger counting. Employees could often be seen at bus stops or train stations with a clipboard, tracking the flow of commuters throughout the day, often at multiple stops along a single line. This allowed the recorded numbers to be imported into spreadsheet software, resulting in charts that show passenger density or even their distribution according to certain parameters.

Despite its limited scope, this approach has a few advantages. Data is rarely faulty as there is nothing in the loop that could technically malfunction. By hand-picking the numbers to record, it's possible to analyse an interchange in-depth; for example, to find the number of people boarding trains after leaving a specific underpass exit. More specific data can also be recorded about transit users, such as their age, gender and carried luggage.

The downsides however, are easy to see. The ongoing costs may be high due to employee compensation. It's also nearly impossible to cover a sufficiently large percentage of the system, lots of extrapolation is needed, which affects accuracy. Due to these factors, in-person counting has taken on an auxiliary role, only used as an additional measure or in technically challenging situations.

\subsection{Physical counting}

In 1983, the Budapest Transport Privately Held Corporation (BKV Zrt.) purchased simple computers from the Austrian Knorr-Bremse company, capable of measuring and interpreting the pressure placed on a vehicle's hydraulic system.\cite{ik260_article} With known standard values, the current passenger count could be guessed with decent precision when travelling between any two stops. The numbers could then be used when planning new timetable assignments -- to increase or decrease headway between subsequent vehicles. With real computational power at a premium, this was a creative solution for the time.

Road vehicles, including cars and bicycles can be tallied using inductive loop counters.\cite{fhwaLoop} In use since the 1960s, these are simple circuits including an insulated wire placed into the road surface by sawing or otherwise cutting the asphalt. Whenever a vehicle drives over the wire, the inductivity changes. By attaching a simple processor, the frequency and length of signals can be mapped to different vehicle classes -- a semi truck and family sedan have entirely different signatures. 

Due to their simplicity and reliability, they remain widely used in many countries. Budapest has around seven hundred induction loops that give a comprehensive, near real-time picture of the city's traffic.\cite{BkkDataCollection}

\subsection{Digital counting}
Storing public traffic camera pictures can reveal a lot about a junction's dominant flow direction. Picture analysis is an extensive subfield, data sanitisation, object detection and machine inference is highly applicable here. As an added feature, the cameras can be used to catch traffic offenders and dangerous drivers quickly.

Many cities around the world currently use automated passenger counting to inform decisions. Examples include London (TfL), Berlin (BVG) and Chicago (CTA). The usual solution includes infrared sensors mounted above the doors of certain vehicles. These detectors can count boarding and disembarking people in real time, and upload this information to a central database. Aside from the already mentioned purpose of timetable planning, this data can be used to handle overcrowding events -- such as a football match or live show -- by deploying more vehicles to increase capacity. As the devices are hooked into the vehicle's onboard computer, the numbers can be correlated with the location and time. CCTV cameras often act as a secondary source or means of verification for these schemes.

With the rise of online and mobile journey planner applications, users themselves can be the source of statistics. Multiple passengers searching for the same destination, or the use of ticketing and geolocation functions within an app can be inputs to feed into already used models. Built-in reporting systems, such as Google's analytics (available on Firebase) help spread this practice.

\section{City modelling}

Processing usage numbers is usually a fairly conventional affair (using spreadsheets, graphing the data, or utilising some form of machine learning), but designing roads, transport lines and vehicles that fulfil the needs of a city requires specialist software. I'll introduce the professional tool and the video game that inspired my work.

"Cities: Skylines" is a single-player city-builder game developed by Colossal Order and published by Paradox Interactive in 2015. The game is a spiritual successor to the SimCity franchise, discontinued in 2013. The player's task is to build a town and metropolitan area from the ground up, requiring a decent sense of economics. The game is known for its detailed and at times challenging to render graphics on any zoom level. Cities are divided into residential, commercial, industrial and office zones, denoted with the now traditional colour scheme of green-blue-yellow. A sequel was released in 2023.

While not intended for professional use, the simulation system, coupled with countless available community modifications make it a strong contender for quick iterations on real life cities. Each member of the population constantly makes decisions on travelling and their living situation, moving homes when it's advisable. Traffic management is very important, commercial buildings require lots of freight trucks in order to get goods, for example This mirrors real-life zoning concepts and forces a hierarchical road system to be built. Since I've played the game for over 100 hours, it's what influenced the project the most. I made use of the zoning colours and incorporated individual building information pages, as well as a version of the game's camera system. Initially, it was a full recreation of the smoothened three-dimensional camera, but I ended up going for a top-down view, as this looked much better with the final heatmap.

PTV Visum is a more involved, planning-focused software used by transit agencies, such as the Swiss Federal Railways. The app merges real socio-economic data and highly detailed transit models to predict trends. Cost analysis and fleet planning is also supported. It's well known for a very high level of detail, but this means sacrifices in performance and the user interface, which mostly reminds me of the 2012 version of Microsoft's Visual Studio. The maps are nevertheless highly detailed, with nodes and links that act similarly to OpenStreetMap's data. Visum places a strong emphasis on traffic simulation, with custom scenarios that can be compared to real trends easily. Multimodal (such as park and ride) travel is also supported. The main idea I used was the deferred evaluation of assignments; my heatmaps are calculated similarly, with only the result being shown. This is less visually appealing but overall provides more information.