\chapter{Goals and requirements}

The goal of this project is to provide a three-dimensional map using open source data, and to create a visualisation tool on top of that. The city's buildings should be distinguishable, both with their shapes as well as available metadata, such as addresses or year of construction. The user should be able to move around the viewpoint, zoom in or out, and select buildings to inspect them. The software should be able to run a simulation of inhabitants and show the most popular areas in the city. It must be possible to load different cities, but only one needs to be active at a time.

The app shall be divided into layers that resemble a usual server-client architechture. Communication should be done using standard public protocols, but data does not need to be published to arbitrary HTTP clients. The client should not hold more information than is required for the visualisation; computation-heavy tasks need to be offloaded to the server. The graphical capabilities of the app shall not have hardware-specific requirements, such as Nvidia's RTX. The layers should be able to run on the same computer, or entirely different machines too. The program should be able to run on the Windows, Linux and Mac desktop operating systems.

The user interface must be understandable by people without a technical background, it should give feedback for actions taken. The technologies used should be maintainable and up to date. The software should also be distributable in a containerised form, for a zero-friction installation.