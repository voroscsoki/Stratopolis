%----------------------------------------------------------------------------
\chapter{Technologies}

I worked using tools from the JVM\footnote{JVM: \@Java Virtual Machine, a fully specified runtime currently owned by Oracle, mainly intended to run programs written in Java} ecosystem, focusing on the problem description's key values, modernity, compatibility and portability. As such, I avoided using old versions and unstable dependencies wherever possible.

\section{Kotlin}

Kotlin is a multiplatform, statically typed, multi-paradigm programming language. Its specification is publicly accessible and updated regularly by JetBrains employees.~\cite{KotlinSpec} The first prototype was released in 2011, and the code was open sourced just a year later. The first full release of the language was published in 2016; since 2017, Android officially supports it, and Android development has been Kotlin-first since 2019.~\cite{KotlinPast}

Most of its popularity can be attributed to its intuitive syntax and near-perfect compatibility with JVM. Any Java code can be referenced in a Kotlin program, allowing the use of countless Java community libraries. There is no restriction on the ratio of source files written in the two languages. It is worth noting however, that this compatibility is unidirectional -- Java code can't call Kotlin code.

Finished programs go through compile and optimisation phases before they are turned into executable applications or libraries. Binaries made with Kotlin's compiler have performance comparable to their traditional JVM counterparts. There are non-JVM compile targets as well:
\begin{itemize}
    \item Kotlin/Native (Stable since 2023, provides native solutions for many operating systems without the need for a virtual machine)
    \item Kotlin/JS (programs are compiled to JavaScript, creating an interpreted codebase from compile-ready one; ideal for web browsers)
\end{itemize}

The language's code structure is simple, easy to understand and quick to write. A lot of the frustrating syntactic and technical elements from Java were improved or built around. For example, the built-in null\footnote{null: a value that specifically denotes missing information; in Java, any object can have a null value} safety system can prevent many runtime errors and eliminates the need for superfluous null checks. Java libraries often have @NonNull or @Nullable annotations to inform the programmer, the Kotlin compiler can translate these. Type inference is also a strong point, as the compiler can guess the class used in all but the most complicated genericised statements. This improves legibility and can help write type-safe code. As well as object-oriented programming, the functional paradigm is well supported and encouraged by the use of lambda functions and pipeline-based list processing.


I used the newest (2.0.21) version for my thesis through JetBrains' own integrated developer environment, IntelliJ Idea. I targeted the Oracle OpenJDK distribution's 21st main version, and compatibility with this LTS\footnote{Long Term Support, in software development, a publisher's pledge to provide longer-lasting maintenance for a specific release} version is guaranteed. JDK 21 was released in September 2023and will lose support in 2028.~\cite{JavaRoadmap}

\section{OpenGL}

Popular low-level graphical library, capable of addressing the current screen down to the pixel level. It was first published in 1992, the newest version is 4.6, dating back to 2017 -- this is the single exception from my goals of using the latest tech.~\cite{OpenglHistory} OpenGL can be praised for its robustness and widespread support, the necessary graphics functions can be accessed on any operating system and by using nearly any programming language. Applications following the OpenGL standard can be expected to look identical on any system running it, provided the necessary dependencies are installed.

\section{libGDX}

LibGDX got the name from its multiple use cases: it's a "game development" and "effects" library for Java-based projects. It provides helper functions to make OpenGL operations callable in a JVM-idiomatic way, making it surprisingly easy to adjust graphics code to the style of the rest of the application.

Its predecessor was meant to aid the display of Android applications specifically. In 2010, the code was open sourced, creating the community-focused development of today.~\cite{LibgdxHistory} The current version number is 1.13.0. Created games and apps can be presented on desktop environments (Windows, macOS, Linux), mobile operating systems (Android, iOS) and even in most web browsers.

\section{Exposed}

Databases, specifically SQL\footnote{Structured Query Language, a domain-specific language used in the management of relational databases} have very old roots. Although the past decade saw many improvements -- mainly with the release of developer-friendly tools -- the standard's 40 years long history still strongly affects present-day programs.


The language's dialects are similar to regional accents, in a linguistic sense; a "speaker" of one variant may understand most or all other types, but that doesn't mean that they have the ability to correctly "express" themselves when using other types. There are multitudes of ORM\footnote{Object Relational Mapping, a technique intended to transform database records and in-memory objects into one another} libraries intended to bridge this gap. The most well-known ones include Hibernate for Java, EntityFramework for \.NET/\texttt{C\#}, and Active Record used in Ruby on Rails applications. The current preferred Kotlin solution is Exposed, built mainly on JDBC (Java Database Connectivity). The library reflects Kotlin's philosophy of developer freedom -- data can be accessed in a classic object-oriented manner, or through a custom language highly reminiscent of raw SQL.~\cite{ExposedDocs}


Portability is ensured by database-agnosticism. Switching between supported database engines only requires changing a few configuration values -- this allows releasing apps with one of many implementations, from SQLite (popular on mobile) to Apache Cassandra (used by many enterprise applications and intended for use with replication between multiple data centers).

\section{Ktor}

Ktor is a Kotlin-based framework for creating asynchronous web applications or services. JetBrains is the primary developer since 2018, with community contributions encouraged. The bulk of this library is the recreation of HTTP and Representational State Transfer (REST) concepts in a Kotlin environment, strongly leveraging coroutines and the language's own syntax.~\cite{KtorDocs} It's currently on its third main version, released in 2024. Its dependencies include the Netty non-blocking I/O server, which works well with asynchronous operations. Jetty or other compliant hosts may also be used. At minimum Java 8 is required for it, and all major operating systems are supported; this means a mobile app may use this code to access RESTful resources over the internet.



\section{Gradle}

Gradle is a build automation tool, making JVM and C++ programs far easier to create, and bundle for any system.\cite{GradleDocs} First released in 2008, the ninth main version is in the works currently, with features and improvements added every year. Most IDEs have built-in support for the tool, making it a favourite for IntelliJ Idea and Android Studio users alongside Maven, another packaging and build tool. Scripts may be written in Groovy -- another Java-compatible JVM language -- or Kotlin, using a Domain Specific Language (DSL) to define individual tasks and required dependencies. Gradle can also be leveraged to write documentation, run tests, or automate software deployment.