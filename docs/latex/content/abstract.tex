\pagenumbering{roman}
\setcounter{page}{1}

\selecthungarian

%----------------------------------------------------------------------------
% Abstract in Hungarian
%----------------------------------------------------------------------------
\chapter*{Kivonat}\addcontentsline{toc}{chapter}{Kivonat}

Az egyre növekvő városiasodás miatt minden eddiginél nagyobb az igény a jól megtervezett tömegközlekedési rendszerekre és úthálózatokra. Azonban sok, a releváns adatgyűjtést és adatelemzést támogató szoftveres eszköz lassú, körülményesen használható vagy idejétmúlt.

Vizualizációs programot készítettem, amivel a felhasználó egy nyílt adatforrásból épített várostérképet szemlélhet, az egyes épületekről részletesebb információt szerezve. Ezen felül statisztikai alapú szimuláció is futtatható, a város lakóinak közlekedési szokásait lehet vele hőtérképen ábrázolni. A szoftver szerver és kliens részekre van bontva, és modern technológiák használatával biztosít rugalmas platformot a jövőbeli fejlesztésekhez. A felhasználói felület könnyen kezelhető, a program támogatja a legnépszerűbb asztali operációs rendszereket. A szerver Docker konténerként is futtatható.


\vfill
\selectenglish


%----------------------------------------------------------------------------
% Abstract in English
%----------------------------------------------------------------------------
\chapter*{Abstract}\addcontentsline{toc}{chapter}{Abstract}

Due to ever-increasing urbanisation, the need for well-designed transit systems and road networks has never been higher. However, many software tools that support the relevant data collection and analysis are slow, difficult to use or otherwise outdated. 

I created a visualisation program that lets the user inspect a map of a city, created from open data. Individual buildings can be inspected, and a statistical simulation can be run to see a heatmap based on the inhabitants' travel patterns. The software is split into a server and a client, and uses modern technologies to serve as a flexible platform for future development. The user interface is designed to be easy to use, and the program is compatible with all major desktop operating systems. The server can also be run as a Docker container.




\vfill
\selectthesislanguage

\newcounter{romanPage}
\setcounter{romanPage}{\value{page}}
\stepcounter{romanPage}