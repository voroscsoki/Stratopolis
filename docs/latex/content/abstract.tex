\pagenumbering{roman}
\setcounter{page}{1}

\selecthungarian

%----------------------------------------------------------------------------
% Abstract in Hungarian
%----------------------------------------------------------------------------
\chapter*{Kivonat}\addcontentsline{toc}{chapter}{Kivonat}

Az egyre növekvő városiasodás miatt minden eddiginél nagyobb az igény a jól megtervezett tömegközlekedési rendszerekre és úthálózatokra. A hatékony várostervezéshez és forgalomszabályozáshoz a lakosságra jellemző adatok hatékony beszerzése és feldolgozása kulcsfontosságú, azonban sok, a releváns adatgyűjtést és adatelemzést támogató szoftveres eszköz lassú, körülményesen használható vagy idejétmúlt, és nem felel meg napjaink elvárásainak.

Témám keretein belül adatvizualizációs programot készítettem, amivel a felhasználó egy nyílt adatforrásból épített várostérképet szemlélhet, az egyes épületekről részletesebb információt szerezhet. Ezen felül ágensalapú szimuláció is futtatható, a város lakóinak közlekedési szokásait lehet vele hőtérképen ábrázolni. A szoftver szerver és kliens részekre van bontva, és modern technológiák használatával biztosít rugalmas platformot a jövőbeli fejlesztésekhez. A felhasználói felület könnyen kezelhető és a program támogatja a legnépszerűbb asztali operációs rendszereket. A szerver Docker konténerként is futtatható tetszőleges kiszolgáló gépen, ezzel csökkentve a felhasználó számítógépét érő terhelést.

A szakdolgozatban bemutatom a felhasznált adatforrást, a projekt felépítését, működését, a felmerült szakterületi és grafikai kihívásokat. A dokumentum kitér a szoftver tesztelésére és telepítési lehetőségeire, valamint felvázolja a potenciális továbbfejlesztési irányokat.



\vfill
\selectenglish


%----------------------------------------------------------------------------
% Abstract in English
%----------------------------------------------------------------------------
\chapter*{Abstract}\addcontentsline{toc}{chapter}{Abstract}

Due to ever-increasing urbanisation, the need for well-designed transit systems and road networks has never been higher. The collection and processing of data that characterises the population is crucial for effective urban planning and traffic control. However, many software tools that support the relevant data collection and analysis are slow, difficult to use or otherwise outdated, failing to meet the expectations of experts today.

In this thesis, I created a data visualisation program that lets the user inspect a map of a city, created from open data. Individual buildings can be inspected, and an agent-based simulation can be run to see a heatmap based on the inhabitants' travel patterns. The software is split into a server and a client, and uses modern technologies to serve as a flexible platform for future development. The user interface is designed to be easy to use, and the program is compatible with all major desktop operating systems. The server can also be run as a Docker container on a dedicated host machine to lessen the workload on the user's computer.

The thesis presents the used data source, the structure and operation of the project, as well as the graphical and transit-specific challenges encountered. It also discusses testing and deployment of the software, and outlines possible future improvements.






\vfill
\selectthesislanguage

\newcounter{romanPage}
\setcounter{romanPage}{\value{page}}
\stepcounter{romanPage}